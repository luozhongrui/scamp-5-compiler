The culmination of the compilation process is the generation of SCAMP-5 assembly code. This phase entails traversing the semantically validated AST and meticulously converting each node into its corresponding assembly instruction. The \texttt{AssemblyGenerator} class spearheads this conversion, methodically appending each generated instruction to an internal list, ensuring a coherent sequence of operations \cite{cooper2012engineering}.

The resulting assembly code serves as a direct translation of the operations defined in the source code, rendered into low-level instructions capable of execution on the SCAMP-5 hardware. This translated code is instrumental for running simulations on the SCAMP-5 simulator, allowing users to extensively test and validate their algorithms within a controlled environment prior to deployment on the actual vision chip.

For example, an assignment such as \texttt{scamp5.A = scamp5.B.north\_west()} is adeptly translated into a series of SCAMP-5 assembly instructions that facilitate the movement of data from register \texttt{B} to \texttt{A} in the northwest direction, employing the designated low-level instruction (\texttt{mov2x}).

This comprehensive process, spanning from AST generation through semantic analysis to final code generation, is meticulously designed to ensure a seamless translation from high-level Python code to efficient SCAMP-5 assembly code. This streamlined methodology significantly enhances the rapid development and testing of image processing algorithms, ensuring they are finely optimized for execution on the SCAMP-5 hardware.
